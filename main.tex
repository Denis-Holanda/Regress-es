\documentclass{article}

% Language setting
% Replace `english' with e.g. `spanish' to change the document language
\usepackage[english]{babel}

% Set page size and margins
% Replace `letterpaper' with `a4paper' for UK/EU standard size
\usepackage[letterpaper,top=2cm,bottom=2cm,left=3cm,right=3cm,marginparwidth=1.75cm]{geometry}

% Useful packages
\usepackage{amsmath}
\usepackage{graphicx}
\usepackage[colorlinks=true, allcolors=blue]{hyperref}



\begin{document}
\date{} % Remove a data padrão
\begin{titlepage}
    \begin{center}
        \LARGE FATEC BAIXADA SANTISTA \\ Curso de Ciência de Dados\\
        
        \vspace{8cm}
        
        \LARGE Relatório: Utilizando de Regressões Lineares e Logísticas em um Dataset Salarial para o Treinamento de uma Rede Neural\\
        
        \vspace{8cm}
        
        \large Autores: João Vitor Ferreira Lima\\
        Denis Holanda Medeiros de Araújo \\
Professor Orientador: Alexandre Garcia de Oliveira \\
        
        \vspace{2cm}
        
        \large Data: 28/11/2024
    \end{center}
\end{titlepage}
\maketitle


\section{Introdução}

No seguinte relatório, será apresentada uma análise utilizando Regressões Lineares aplicada a um conjunto de dados ficcionais de salários, provindo do site da Kaggle. O objetivo é realizar de forma prática os conceitos vistos em aula.

\section{Descrição do Conjunto de Dados}

O conjunto de dados que está sendo utilizado neste estudo é o "Salary-Data", obtido a partir do site Kaggle. Contém informações fictícias de idade, gênero, escolaridade, cargo e anos de experiência de funcionário.

\section{Dimensão dos Dados}

O conjunto de dados possui 6.704 instâncias (amostras) que foram contabilizadas para a criação deste relatório, com o número de amostras sendo reduzidas a 12 para a análise, o que inclui idade e salário como principais métricas.


\section{Visualização das primeiras linhas do conjunto de dados}
A figura abaixo são os dados antes da normalização e tratamento dos valores ausentes.
\begin{figure}
\centering
\includegraphics[width=0.8\linewidth]{tabela.png}
\caption{\label{fig:frog}Visualização das primeiras linhas antes do tratamento pelo Google Colab}
\end{figure}

\newpage
\subsection{Regressão Linear}

A Regressão Linear é uma técnica estatística e de aprendizado de máquina usada para modelar a relação entre uma variável dependente (ou variável de resposta) e uma ou mais variáveis independentes (ou preditoras). Seu objetivo principal é encontrar a linha (ou plano, em dimensões maiores) que melhor representa essa relação, de forma que seja possível prever ou estimar valores futuros da variável dependente com base nas variáveis independentes.
Na página abaixo está um exemplo de regressão linear aplicada ao nosso dataset.
\begin{figure}
    \centering
    \includegraphics[width=0.8\linewidth]{Regressão Linear.png}
    \caption{\label{fig:frog}Gráfico da regressão linear do dataset}
\end{figure}

\subsection{Regressão Logistica}

A regressão logística transforma a saída de uma equação linear em uma probabilidade entre 0 e 1 usando a função sigmoide. Essa probabilidade pode então ser usada para classificar os dados em diferentes categorias (como 0 ou 1).
Na página abaixo está um exemplo de Regressão Logística aplicada ao nosso dataset.

\begin{figure}
    \centering
    \includegraphics[width=0.8\linewidth]{Regressão Logistica.png}
    \caption{\label{fig:frog}Gráfico da regressão linear do dataset}
\end{figure}

\subsection{Rede Neural}

Uma Rede Neural Artificial (RNA) é um modelo computacional composta por uma rede de nós organizados em camadas que processam informações, ajustando-se de forma iterativa para resolver problemas complexos de aprendizado, como classificação, regressão e geração de dados, nós usaremos as informações apresentadas anteriormente para treinar a rede neural.
Na página abaixo está um exemplo de um Treinamento de Rede Neural ao nosso dataset.

\begin{figure}
    \centering
    \includegraphics[width=0.5\linewidth]{Treinamento da Rede Neural.png}
    \caption{\label{fig:frog}Gráfico do treinamento da rede neural do dataset}
\end{figure}


\newpage
\section{Resultados}

Podemos analisar que, enquanto o teste de regressão linear mostra uma discrepância ascendente que comprova o crescente aumento médio do salário em correlação à idade de cada trabalhador, ao analisarmos da regressão logística vemos que os dados são inconclusivos para uma leitura precisa se tentarmos ler do ponto de vista da reta.


\pagebreak


\bibliographystyle{alpha}
\bibliography{sample}
O código-fonte do projeto está disponível no GitHub do seguinte link:\\ \url{https://github.com/Denis-Holanda/Regress-es/blob/main/Regressoes.ipynb}.\\
O link para o dataset usado neste relatório:\\ \url{https://www.kaggle.com/datasets/mohithsairamreddy/salary-data}.
\end{document}
